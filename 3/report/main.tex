\documentclass{article}

\usepackage{graphicx} % Required for the inclusion of images
\usepackage{amsmath} % Required for some math elements
\usepackage{hyperref}
\usepackage[numbers,sort]{natbib}
\usepackage[russian,english]{babel}
\usepackage[T2A]{fontenc}
\usepackage{placeins}
\usepackage{hyperref}
\usepackage[figurename=Рисунок]{caption}
\graphicspath{ {./figs/} }
\setlength\parindent{0pt} % Removes all indentation from paragraphs
\renewcommand{\labelenumi}{\alph{enumi}.} % Make numbering in the enumerate environment by letter
\title{Адаптивное и робастное управление \\ Работа №3 В-17\\ Отчет} % Title
\author{Кирилл Лалаянц \\ Прокопов Егор} % Author name
\date{\today} % Date for the report

\begin{document}

\maketitle % Insert the title, author and date

\begin{center}
\begin{tabular}{l r}
%Date Performed: & January 1, 2012 \\ % Date the experiment was performed
% Выполнили: & Кирилл Лалаянц \\ % Partner names
% &Прокопов Егор \\
Преподаватель: & Козачёк О.А. % Instructor/supervisor
\end{tabular}
\end{center}
\newpage
% \begin{abstract}
% Abstract text
% \end{abstract}

\section{Цель работы}
Освоение принципов построения адаптивной системы управления многомерным объектом

\section{Данные по варианту}
Матрица $A =\begin{bmatrix} 0 & 1 \\ -1 & 7\end{bmatrix}$, время переходного процесса $t_n = 0.45$, максимальное перерегулирование $\bar{\sigma} < 15$, сигнал $g(t) = \cos t + 3\sin 2t + 5$.

\section{Выполнение}
\subsection{Формирование эталонной модели}
В этом задании необходимо сформировать эталонную модель

$$
\begin{cases}
\dot x_M = A_Mx_M + b_Mg, \\
y_M = C_Mx_M,
\end{cases}
$$

на основе заданных значений времени переходного процесса $t_n$ и максимального перерегулирования $\bar\sigma$.

По результатам моделирования был построен график переходной функции модели.

\begin{figure}[h!]
  \centering
  \includegraphics[width=0.65\textwidth]{figs/1.png}
  \caption{График переходной функции модели.} 
  \label{fig:1}
\end{figure}

На рис. \ref{fig:1}, что переходная функция эталонной модели соответствует заданным значениям $t_n$ и $\bar\sigma$

\FloatBarrier
\subsection{Построение системы управления при известных параметрах объекта}
В этом задании необходимо промоделировать систему управления объекта с регулятором

$$
\begin{cases}
    \dot x = Ax + bu, \\
    u = \theta^T x + \dfrac{1}{\kappa}g, \\
    y = Cx,
\end{cases}
$$

с предположением, что параметры объекта известны.

В ходе моделирования было проведено три эксперимента. В первом эксперименте использовались рассчетные значения параметров объекта, заложенные в $\theta_1 = -5.07$ и $\theta_2 = -2.01$. Результаты моделирования приведены на рис. \ref{fig:2_ideal_x} и \ref{fig:2_ideal_e}. Видно, что система следит за сигналом с минимальными отклонениями, являющимися лишь ошибками вычислений.

\begin{figure}[h!]
  \centering
  \includegraphics[width=0.65\textwidth]{figs/2_ideal_x.png}
  \caption{Графики траекторий $x_M(t)$ и $x(t)$ первого эксперимента.} 
  \label{fig:2_ideal_x}
\end{figure}

\begin{figure}[h!]
  \centering
  \includegraphics[width=0.65\textwidth]{figs/2_ideal_e.png}
  \caption{График ошибки $e(t)$ первого эксперимента.} 
  \label{fig:2_ideal_e}
\end{figure}
\FloatBarrier

В ходе второго экспримента параметры объекта были незначительно отклонены, $A_{new} =\begin{bmatrix} 1 & 2 \\ 0 & 8\end{bmatrix}$. Результаты моделирования приведены на рис. \ref{fig:2_not_ideal_1.1_x} и \ref{fig:2_not_ideal_1.1_e}. Видно, что система остается устойчивой, однако ошибка значительно возросла по сравнению с первым экспериментом.

\begin{figure}[h!]
  \centering
  \includegraphics[width=0.65\textwidth]{figs/2_not_ideal_1.1_x.png}
  \caption{Графики траекторий $x_M(t)$ и $x(t)$ второго эксперимента.} 
  \label{fig:2_not_ideal_1.1_x}
\end{figure}

\begin{figure}[h!]
  \centering
  \includegraphics[width=0.65\textwidth]{figs/2_not_ideal_1.1_e.png}
  \caption{График ошибки $e(t)$ второго эксперимента.} 
  \label{fig:2_not_ideal_1.1_e}
\end{figure}
\FloatBarrier

В ходе третьего эксперимента параметры объекта были отклонены значительно, $A_{new} =\begin{bmatrix} 5 & 6 \\ 4 & 12\end{bmatrix}$.Результаты моделирования приведены на рис. \ref{fig:2_bad_5_x} и \ref{fig:2_bad_5_e}. Видно, что система неустойчива, а ошибка неограниченно растет.

\begin{figure}[h!]
  \centering
  \includegraphics[width=0.65\textwidth]{figs/2_bad_5_x.png}
  \caption{Графики траекторий $x_M(t)$ и $x(t)$ третьего эксперимента.} 
  \label{fig:2_bad_5_x}
\end{figure}

\begin{figure}[h!]
  \centering
  \includegraphics[width=0.65\textwidth]{figs/2_bad_5_e.png}
  \caption{График ошибки $e(t)$ третьего эксперимента.} 
  \label{fig:2_bad_5_e}
\end{figure}
\FloatBarrier

\FloatBarrier
\subsection{Построение адаптивной системы управления}
В этом задании необходимо промоделировать адаптивную систему управления объекта с регулятором

$$
\begin{cases}
    \dot x = Ax + bu, \\
    y = Cx, \\
    u = \hat\theta^T x + \dfrac{1}{\kappa}g, \\
    \dot{\hat \theta} = \gamma x b^T P e,
\end{cases}
$$

В ходе моделирования было проведено три эксперимента для фиксированного значения $\gamma = 1000$. В ходе первого эксперимента параметры не были отклонены. Результаты моделирования представлены на рис. \ref{fig:3_1_0_x} и \ref{fig:3_1_0_e}. Видно, что система сходится, а ошибка стремится к нулю.

\begin{figure}[h!]
  \centering
  \includegraphics[width=0.65\textwidth]{figs/3_1_0_x.png}
  \caption{Графики траекторий $x_M(t)$ и $x(t)$ первого эксперимента адаптивной системы.} 
  \label{fig:3_1_0_x}
\end{figure}

\begin{figure}[h!]
  \centering
  \includegraphics[width=0.65\textwidth]{figs/3_1_0_e.png}
  \caption{График ошибки $e(t)$ первого эксперимента адаптивной системы.} 
  \label{fig:3_1_0_e}
\end{figure}
\FloatBarrier

В ходе второго экспримента параметры объекта были незначительно отклонены, $A_{new} =\begin{bmatrix} 1 & 2 \\ 0 & 8\end{bmatrix}$. Результаты моделирования приведены на рис. \ref{fig:3_1_1_x} и \ref{fig:3_1_1_e}. Видно, что система остается устойчивой, а ошибка лишь незначительно возросла по сравнению с первым экспериментом.

\begin{figure}[h!]
  \centering
  \includegraphics[width=0.65\textwidth]{figs/3_1_1_x.png}
  \caption{Графики траекторий $x_M(t)$ и $x(t)$ второго эксперимента адаптивной системы.} 
  \label{fig:3_1_1_x}
\end{figure}

\begin{figure}[h!]
  \centering
  \includegraphics[width=0.65\textwidth]{figs/3_1_1_e.png}
  \caption{График ошибки $e(t)$ второго эксперимента адаптивной системы.} 
  \label{fig:3_1_1_e}
\end{figure}
\FloatBarrier

В ходе третьего эксперимента параметры объекта были отклонены значительно, $A_{new} =\begin{bmatrix} 5 & 6 \\ 4 & 12\end{bmatrix}$. Результаты моделирования приведены на рис. \ref{fig:3_1_5_x} и \ref{fig:3_1_5_e}. Видно, что система устойчива, а ошибка стремится к нулю.

\begin{figure}[h!]
  \centering
  \includegraphics[width=0.65\textwidth]{figs/3_1_5_x.png}
  \caption{Графики траекторий $x_M(t)$ и $x(t)$ третьего эксперимента адаптивной системы.} 
  \label{fig:3_1_5_x}
\end{figure}

\begin{figure}[h!]
  \centering
  \includegraphics[width=0.65\textwidth]{figs/3_1_5_e.png}
  \caption{График ошибки $e(t)$ третьего эксперимента адаптивной системы.} 
  \label{fig:3_1_5_e}
\end{figure}
\FloatBarrier

Из всех трех экспериментов видно, что насколько сильным не было бы отклонение параметров объекта, благодаря адаптивному закону управления система остается устойчивой.

Используя расчетные значения параметров объекта, были проведены эксперименты с тремя различными значениями $\gamma = \{0.0001, 1, 20\}$. Результаты моделирования представлены на рис. \ref{fig:3_2_0.0001_x}-\ref{fig:3_2_20_e}.

\begin{figure}[h!]
  \centering
  \includegraphics[width=0.65\textwidth]{figs/3_2_0.0001_x.png}
  \caption{Графики траекторий $x_M(t)$ и $x(t)$ адаптивной системы при $\gamma=0.0001$.} 
  \label{fig:3_2_0.0001_x}
\end{figure}

\begin{figure}[h!]
  \centering
  \includegraphics[width=0.65\textwidth]{figs/3_2_0.0001_e.png}
  \caption{График ошибки $e(t)$ адаптивной системы при $\gamma=0.0001$.} 
  \label{fig:3_2_0.0001_e}
\end{figure}

\begin{figure}[h!]
  \centering
  \includegraphics[width=0.65\textwidth]{figs/3_2_1_x.png}
  \caption{Графики траекторий $x_M(t)$ и $x(t)$ адаптивной системы при $\gamma=1$.} 
  \label{fig:3_2_1_x}
\end{figure}

\begin{figure}[h!]
  \centering
  \includegraphics[width=0.65\textwidth]{figs/3_2_1_e.png}
  \caption{График ошибки $e(t)$ адаптивной системы при $\gamma=1$.} 
  \label{fig:3_2_1_e}
\end{figure}

\begin{figure}[h!]
  \centering
  \includegraphics[width=0.65\textwidth]{figs/3_2_20_x.png}
  \caption{Графики траекторий $x_M(t)$ и $x(t)$ адаптивной системы при $\gamma=20$.} 
  \label{fig:3_2_20_x}
\end{figure}

\begin{figure}[h!]
  \centering
  \includegraphics[width=0.65\textwidth]{figs/3_2_20_e.png}
  \caption{График ошибки $e(t)$ адаптивной системы при $\gamma=20$.} 
  \label{fig:3_2_20_e}
\end{figure}
\FloatBarrier

В конце был проведен эксперимент с $\gamma=1000$, $A_{new} =\begin{bmatrix} 5 & 6 \\ 4 & 12\end{bmatrix}$ при $g(t) = 1$. Результаты моделирования представлены на рис.\ref{fig:3_3_5_x} и \ref{fig:3_3_5_e}.

\begin{figure}[h!]
  \centering
  \includegraphics[width=0.65\textwidth]{figs/3_3_5_x.png}
  \caption{График траекторий $x_M(t)$ и $x(t)$ адаптивной системы при $g(t) = 1$.} 
  \label{fig:3_3_5_x}
\end{figure}

\begin{figure}[h!]
  \centering
  \includegraphics[width=0.65\textwidth]{figs/3_3_5_e.png}
  \caption{График ошибки $e(t)$ адаптивной системы при $g(t) = 1$.} 
  \label{fig:3_3_5_e}
\end{figure}

\FloatBarrier
\section{Заключение}

\end{document}
