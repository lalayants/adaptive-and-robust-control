\documentclass{article}

\usepackage{graphicx} % Required for the inclusion of images
\usepackage{amsmath} % Required for some math elements
\usepackage{hyperref}
\usepackage[numbers,sort]{natbib}
\usepackage[russian,english]{babel}
\usepackage[T2A]{fontenc}
\usepackage{placeins}
\usepackage{hyperref}
\usepackage[figurename=Рисунок]{caption}
\usepackage{ dsfont }
\graphicspath{ {./figs/} }
\setlength\parindent{0pt} % Removes all indentation from paragraphs
\renewcommand{\labelenumi}{\alph{enumi}.} % Make numbering in the enumerate environment by letter
\title{Адаптивное и робастное управление \\ Курсовая работа\\ В-1} % Title
\author{Кирилл Лалаянц} % Author name
\date{\today} % Date for the report

\begin{document}

\maketitle % Insert the title, author and date

\begin{center}
\begin{tabular}{l r}
%Date Performed: & January 1, 2012 \\ % Date the experiment was performed
% Выполнили: & Кирилл Лалаянц \\ % Partner names
% &Прокопов Егор \\
Преподаватель: & Козачёк О.А. % Instructor/supervisor
\end{tabular}
\end{center}
\newpage
% \begin{abstract}
% Abstract text
% \end{abstract}

\section{Цель работы}

Синтез закона адаптивного управления с использованием метода расширенной ошибки и схемы Лайона, обеспечивающего
ограниченность всех сигналов и слежение выхода объекта за эталонным
сигналом так, чтобы:
\[\lim_{t \rightarrow \infty}(y_M(t) - y(t)) = 0\]
Выход эталонного объекта \(y_M(t)\):
\[y_M(t) = \frac{1}{K_M(s)}[g(t)]=  \frac{1}{s^2 + 5s + 6}[g(t)]\]
Выход объекта \(y(t)\):
\[y(t) = \frac{b_0}{s^2 + a_1 s + a_0}[u]\]

Вариант 1: 
\begin{itemize}
  \item \(a_1 = 2\); \(a_0 = -3\);
  \item \(b_0 = 2\);
  \item \(g(t) = 7 \cos(3t + 2) + 8\)
\end{itemize}


\section{Выполнение}
\subsection{Проверка объекта управления на свойства полной управляемости и наблюдаемости.}
\[U = \begin{pmatrix}
  1 & -2 \\
  0 & 2
\end{pmatrix} \rightarrow \text{полностью управляемая}\]
\[V = \begin{pmatrix}
  0 & 1 \\
  2 & 0
\end{pmatrix} \rightarrow \text{полностью наблюдаемая}\]

\subsection{Проверка объекта управления на устойчивость и минимально-фазовость}
Для проверки системы на устойчивость найдем полюса знаменателя её передаточной функции \(W(s)\):
\[\text{pole(W(s)) } = [-3, 1] \rightarrow \text{ неустойчивая и не минимально-фазовая}\]

\subsection{Определение и реализация требуемых компонентов системы}
Выберем фильтры:
\[\dot \nu_1 = \Lambda \nu_1 + e_{n-1}u\]
\[\dot \nu_2 = \Lambda \nu_2 + e_{n-1}y\]
гдe \(e_{n-1} = [0 \hdots 1]^T \in \mathds{R}^{n-1}\); \(\nu_1, \nu_2 \in \mathds{R}^{n-1}\).
\[\Lambda = \begin{bmatrix}
  0 & 1 & \hdots & 0 \\
  \vdots & \vdots & \ddots & \vdots \\
  0 & 0 & \hdots & 1 \\
  -k_0 & -k_1 \hdots & k_{n- 2}
\end{bmatrix} \in \mathds{R}^{(n-1) \times (n-1)} \]

Для данного объекта \(n=2\). Следовательно:
\[\dot \nu_1 = -k_0 \nu_1 + u\]
\[\dot \nu_2 = -k_0 \nu_2 + y\]

Зададим вектора:
\[\omega = [\nu_1^T~\nu_2^T~y]^T\]
\[\varphi= -[\omega^T~u]^T\]
\[\bar \varphi= \frac{1}{K_M(s)}[\varphi]\]
\[\hat{\psi}^T = [\tilde \theta ~~ \tilde b_m] \]

Закон управления:
\[u = \frac{1}{\hat b_{M_0}}(-\theta^T \omega + b_{M_0} g(t)) \]

Статическая модель расширенной ошибки:
\[\hat \varepsilon = \bar \varphi^T \tilde \psi =  \varepsilon - \hat{\psi}_p^T \bar \varphi + \frac{1}{K_M(s)}[ \hat{\psi}^T \varphi]\]

Градиентный АА:
\[\dot {\hat{\psi}}_p^T = \gamma \Gamma \frac{\bar \varphi}{1 + \bar \varphi^T \bar \varphi} \hat \varepsilon\]
Модифицированный АА (Схема Лайона):
\[\bar y = \bar \varphi^T \psi\]
\[Y = [\bar y~H_1(s)[\bar y]~H_2(s)[\bar y]~H_3(s)[\bar y]]^T;~~W^T = [\bar \varphi^T~H_1(s)[\bar \varphi^T]~H_2(s)[\bar \varphi^T]~H_3(s)[\bar \varphi^T]]^T\]
\[Y = W \psi\]
\[E = Y - W \psi\]
\[\dot {\hat{\psi}} = \gamma W E\]

Так как в вычислении управления есть деление на \(\hat b_{M_0}\), на практике необходимо реализовать проецирование этого значение на нижнюю границу \(\bar b_0\), в случае если оно становится меньше ее. 
Пусть \(\gamma_0 = diag(1, 1, 1, 1)\), тогда если \(\hat b_{M_0} < \bar b_0\), \(\gamma = \gamma_0 - e_{2n} \times e_{2n}^T = diag(1, 1, 1, 0)\). Это предотвращает дальнейшем уменьшение значения параметра и делает моделирование возможным.

На практике лучше брать последний элемент матрицы \(\gamma_0\) на несколько порядков меньше остальных.
\FloatBarrier

\subsection{Результаты градиентного алгоритма}
На рисунках \ref{fig:1_epsilon}-\ref{fig:1_u} представлены полученные графики. Как видно, существует субоптимальное значение \(\gamma\), увеличение или уменьшение которого приводит к увеличению времени сходимости. Так же заметно, что время сходимости в целом довольно большое.
\begin{figure}[h!]
  \centering
  \includegraphics[width=0.75\textwidth]{tex/images/task1_eps.png}
  \caption{График ошибки слежения линейного объекта с градиентным алгоритмом при различных параметрах \(\gamma\).} 
  \label{fig:1_epsilon}
\end{figure}
\begin{figure}[h!]
  \centering
  \includegraphics[width=0.75\textwidth]{tex/images/task1_psi_p_hat.png}
  \caption{График нормы вектора оценки параметров линейного объекта с градиентным алгоритмом при различных параметрах \(\gamma\).} 
  \label{fig:1_psi_p_hat_hat}
\end{figure}
\begin{figure}[h!]
  \centering
  \includegraphics[width=0.75\textwidth]{tex/images/task1_u.png}
  \caption{График управления линейного объекта с градиентным алгоритмом при различных параметрах \(\gamma\).} 
  \label{fig:1_u}
\end{figure}
\FloatBarrier
\newpage

\subsection{Результаты модифицированного алгоритма}

На рисунках \ref{fig:2_epsilon}-\ref{fig:2_u} представлены полученные графики. В легенде к графику указан параметр \(\gamma_k\). Он использовался для вычисления следующим образом:
\[\gamma_0 = diag(1, 1, 1, 0.01)\]
\[
\gamma = 
\begin{cases}
  \gamma_0\gamma_k,~~~~~\hat b_{M_0} > \bar b_0 \\
  (\gamma_0 - 0.01 e_{4} \times e_{4}^T)\gamma_k,~~~~~\hat b_{M_0} > \bar b_0 \\
\end{cases}\]
Это реализует логику проецирования в коде и предотвращает деление на 0 в случае, если оценка \(\hat b_{M_0}\) стремится к нему.


 Как видно, увеличение \(\gamma\) приводит к уменьшению времени сходимости. Так же заметно, что время сходимости в разы меньше.
\begin{figure}[h!]
  \centering
  \includegraphics[width=0.75\textwidth]{tex/images/task2_eps.png}
  \caption{График ошибки слежения линейного объекта с модифицированным алгоритмом адаптации при различных параметрах \(\gamma\).} 
  \label{fig:2_epsilon}
\end{figure}
\begin{figure}[h!]
  \centering
  \includegraphics[width=0.75\textwidth]{tex/images/task2_psi_p_hat.png}
  \caption{График нормы вектора оценки параметров линейного объекта с модифицированным алгоритмом адаптации при различных параметрах \(\gamma\).} 
  \label{fig:2_psi_p_hat_hat}
\end{figure}
\begin{figure}[h!]
  \centering
  \includegraphics[width=0.75\textwidth]{tex/images/task2_u.png}
  \caption{График управления линейного объекта с модифицированным алгоритмом адаптации при различных параметрах \(\gamma\).} 
  \label{fig:2_u}
\end{figure}

\FloatBarrier
\newpage
\section{Заключение}
В работе было исследовано поведение алгоритма адаптивного управления с использованием метода расширенной ошибки и схемы Лайона. Как видно, модификация действительно приводит к качественному улучшению:
\begin{itemize}
  \item увеличение \(\gamma\) ведет к уменьшению времени сходимости, что позволяет точнее задавать желаемое поведение;
  \item в отличии от градиентного алгоритма, нет оптимального значения, благодаря чему отпадает необходимость его искать;
  \item общее время сходимости уменьшилось в разы;
  \item ошибка слежения стала меньше;
  \item \( \varepsilon \rightarrow 0 \) экспоненциально, если PE;
\end{itemize}
Из недостатком можно выделить только чуть более высокую вычислительную нагрузку.


\end{document}
