\documentclass{article}

\usepackage{graphicx} % Required for the inclusion of images
\usepackage{amsmath} % Required for some math elements
\usepackage{hyperref}
\usepackage[numbers,sort]{natbib}
\usepackage[russian,english]{babel}
\usepackage[T2A]{fontenc}
\usepackage{placeins}
\usepackage{hyperref}
\usepackage[figurename=Рисунок]{caption}
\graphicspath{ {./figs/} }
\setlength\parindent{0pt} % Removes all indentation from paragraphs
\renewcommand{\labelenumi}{\alph{enumi}.} % Make numbering in the enumerate environment by letter
\title{Адаптивное и робастное управление \\ Работа №2 \\ Отчет В-17} % Title
\author{Кирилл Лалаянц \\ Егор Прокопов} % Author name
\date{\today} % Date for the report

\begin{document}

\maketitle % Insert the title, author and date

\begin{center}
\begin{tabular}{l r}
%Date Performed: & January 1, 2012 \\ % Date the experiment was performed
% Выполнили: & Кирилл Лалаянц \\ % Partner names
% &Прокопов Егор \\
Преподаватель: & Козачёк О.А. % Instructor/supervisor
\end{tabular}
\end{center}
\newpage
% \begin{abstract}
% Abstract text
% \end{abstract}

\section{Цель работы}

Освоение процедуры синтеза адаптивного наблюдателя
линейного объекта.

\section{Выполнение}
В работе проведено моделирование системы для \(n = 2\):
\[
\begin{cases}
    \dot x = A x + Bu \\
    y = C x \\
    \hat x = \sum_{i=0}^{n}\hat \theta_{i+1} (sI - A_0)^{-1} e_{n-i} [y] + \sum_{j=0}^{n}\hat \theta_{j + 1 + n} (sI - A_0)^{-1} e_{n - j} [u] \\
    \varepsilon = y - C \hat x  \\ 
    \dot {\hat \theta} = \gamma \omega \varepsilon
\end{cases}
\]
\[
  A = \begin{bmatrix}
    -a_1 && 1\\
    -a_0 && 0
  \end{bmatrix};
  A_0 = \begin{bmatrix}
    -k_1 && 1\\
    -k_0 && 0
  \end{bmatrix};
  B = \begin{bmatrix}
    b_1 \\
    b_0
  \end{bmatrix};
  C = \begin{bmatrix}
    1 && 0
  \end{bmatrix};
\]
\[
A_0 = A - \theta C \text{ -- условие согласования, где } 
\theta ^ T= \begin{bmatrix}
  k_0 - a_0 && k_1 - a_1\\
\end{bmatrix};
\]
\[
  \omega = \begin{bmatrix}
    \frac{1}{K(s)}[y] && \frac{s}{K(s)}[y] && \frac{1}{K(s)}[u] && \frac{s}{K(s)}[u]\\
  \end{bmatrix};
K(s) = s^2 + k_1 s + k_0 .\]

При моделировании использовались следующие значения параметров: 
\begin{itemize}
  \item \(a_1 = 2\); \(a_0 = 4\);
  \item \(b_1 = 3\); \(b_0 = 5\);
  \item \(k_1 = 4\); \(k_0 = 4\);
  \item $\gamma=0.25$ (субоптимальное значение, подобранное экспериментально);
  \item начальные условия всех векторов состояния равны 0.
\end{itemize}

\newpage
\subsection{Управление с одной гармоникой}
Сначала было проведено моделирование при \(u_1 = 10 sin(t)\). 
Так как не выполняется условие незатухающего возбуждения, то параметрическая ошибка \(\tilde \theta = \theta - \hat \theta\) и норма ошибки вектора состояния \(||\hat x - x||\) не сходятся к 0. 
Результаты представлены на рис. \ref{fig:1_theta} - \ref{fig:1_xnorm}.
\begin{figure}[h!]
  \centering
  \includegraphics[width=0.5\textwidth]{figs/3_0_theta.png}
  \caption{График параметрической ошибки адаптивного наблюдателя линейного объекта при управлении \(u_1\).} 
  \label{fig:1_theta}
\end{figure}
\begin{figure}[h!]
  \centering
  \includegraphics[width=0.5\textwidth]{figs/3_0_ex_norm.png}
  \caption{График нормы ошибки вектора состояния адаптивного наблюдателя линейного объекта при управлении \(u_1\).} 
  \label{fig:1_xnorm}
\end{figure}
\FloatBarrier
\newpage

\subsection{Управление с несколькими гармоникой}
Проведено моделирование при \(u_2 = 10 sin(t) + 5 cos(2t) + 4cos(4t) + 3cos(8t)\). 
Так как выполняется условие незатухающего возбуждения, то параметрическая ошибка \(\tilde \theta = \theta - \hat \theta\) и норма ошибки вектора состояния \(||\hat x - x||\) сходятся к 0. 
Результаты представлены на рис. \ref{fig:2_theta} - \ref{fig:2_xnorm}.
\begin{figure}[h!]
  \centering
  \includegraphics[width=0.5\textwidth]{figs/3_1_theta.png}
  \caption{График параметрической ошибки адаптивного наблюдателя линейного объекта при управлении \(u_2\).} 
  \label{fig:2_theta}
\end{figure}
\begin{figure}[h!]
  \centering
  \includegraphics[width=0.5\textwidth]{figs/3_1_ex_norm.png}
  \caption{График нормы ошибки вектора состояния адаптивного наблюдателя линейного объекта при управлении \(u_2\).} 
  \label{fig:2_xnorm}
\end{figure}
\FloatBarrier
\newpage
\section{Заключение}
При соответствующем условию незатухающего возбуждения управлении \(u\), адаптивный наблюдатель линейного объекта имеет следующие свойства:
\begin{itemize}
  \item все сигналы ограничены;
  \item \( \varepsilon \rightarrow 0 \) асимптотически;
  \item \( \tilde \theta \rightarrow 0 \) экспоненциально;
  \item \( x - \hat x \propto \tilde \theta\).
\end{itemize}
Моделирование в среде Simulink подтвердило данные свойства.
\end{document}
